\documentclass[aspectratio=169]{beamer}

\usepackage[ngerman]{babel}
\usepackage[utf8]{inputenc}
\usepackage[T1]{fontenc}
\usepackage{tcolorbox}
\usepackage{algpseudocode}
\usepackage{algorithm}
\usepackage{amsmath}
\usepackage{amsfonts}
\usepackage{amssymb}
\usepackage{amsthm}
\usepackage{graphicx}
\usepackage{enumerate}
\usepackage{xcolor}
\usepackage{tcolorbox}
\usepackage{listings}
\definecolor{mygreen}{RGB}{28,172,0} % color values Red, Green, Blue
\definecolor{mylilas}{RGB}{170,55,241}

\lstset{language=Matlab,%
    %basicstyle=\color{red},
    breaklines=true,%
    morekeywords={matlab2tikz},
    keywordstyle=\color{blue},%
    morekeywords=[2]{1}, keywordstyle=[2]{\color{black}},
    identifierstyle=\color{black},%
    stringstyle=\color{mylilas},
    commentstyle=\color{mygreen},%
    showstringspaces=false,%without this there will be a symbol in the places where there is a space
    numbers=left,%
    numberstyle={\tiny \color{black}},% size of the numbers
    numbersep=9pt, % this defines how far the numbers are from the text
    emph=[1]{for,end,break},emphstyle=[1]\color{red}, %some words to emphasise
    %emph=[2]{word1,word2}, emphstyle=[2]{style},    
}



\usetheme{Hannover}


\title{Projekt: Name}
\author{Ugur Turhal, Silvan Lenzlinger \& Berkan Kurt}
\institute{Universit\"at Basel}
\date{\today}

\begin{document}



\frame{\maketitle}
\frame{\tableofcontents}


\section{Allgemeines} 
\begin{frame}[fragile]
\textbf{ALLGEMEINES}
\begin{columns}
\begin{column}{0.5\textwidth}

\end{column}

\begin{column}{0.5\textwidth}
\begin{lstlisting}

\end{lstlisting}
\end{column}

\end{columns}
\end{frame}



\section{Aufbau der Matrix für Poisson} 
\begin{frame}[fragile]
\textbf{POISSON}
\begin{columns}
\begin{column}{0.5\textwidth}


\end{column}

\begin{column}{0.5\textwidth}

\end{column}

\end{columns}
\end{frame}

\section{Aufbau der Matrix für impl. Euler} 
\begin{frame}[fragile]
\textbf{IMPL. EULER}

\begin{columns}
\begin{column}{0.45\textwidth}


\end{column}
\begin{column}{0.5\textwidth}

\end{column}

\end{columns}
\end{frame}


\section{Aufbau der Matrix für SDIRK2} 
\begin{frame}[fragile]
\textbf{SDIRK2}
\begin{columns}
\begin{column}{0.45\textwidth}

\end{column}
\begin{column}{0.5\textwidth}


\end{column}

\end{columns}
\end{frame}

\end{document}